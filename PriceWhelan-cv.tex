\documentclass[12pt, letterpaper]{apw-cv}

\usepackage{hyperref}
\usepackage{geometry}
\usepackage[T1]{fontenc}
\usepackage{natbib}

% name here
\def\name{\textbf{Adrian M. Price-Whelan}}

% Date formatting.
\usepackage[yyyymmdd]{datetime}
\renewcommand{\dateseparator}{-}

% ADS query link
\def\adsurl{http://adsabs.harvard.edu/cgi-bin/nph-abs_connect?db_key=AST&db_key=PRE&qform=AST&arxiv_sel=astro-ph&arxiv_sel=cond-mat&arxiv_sel=cs&arxiv_sel=gr-qc&arxiv_sel=hep-ex&arxiv_sel=hep-lat&arxiv_sel=hep-ph&arxiv_sel=hep-th&arxiv_sel=math&arxiv_sel=math-ph&arxiv_sel=nlin&arxiv_sel=nucl-ex&arxiv_sel=nucl-th&arxiv_sel=physics&arxiv_sel=quant-ph&arxiv_sel=q-bio&sim_query=YES&ned_query=YES&adsobj_query=YES&aut_logic=OR&obj_logic=OR&author=price-whelan\%2C+Adrian&object=&start_mon=&start_year=&end_mon=&end_year=&ttl_logic=OR&title=&txt_logic=OR&text=&nr_to_return=200&start_nr=1&jou_pick=ALL&ref_stems=&data_and=ALL&group_and=ALL&start_entry_day=&start_entry_mon=&start_entry_year=&end_entry_day=&end_entry_mon=&end_entry_year=&min_score=&sort=SCORE&data_type=SHORT&aut_syn=YES&ttl_syn=YES&txt_syn=YES&aut_wt=1.0&obj_wt=1.0&ttl_wt=0.3&txt_wt=3.0&aut_wgt=YES&obj_wgt=YES&ttl_wgt=YES&txt_wgt=YES&ttl_sco=YES&txt_sco=YES&version=1}

% PDF metadata
\hypersetup{
  colorlinks = true,
  urlcolor = [rgb]{0.1,0.25,0.5},
  pdfauthor = {\name},
  pdfkeywords = {astrophysics, astronomy, physics},
  pdftitle = {\name: Curriculum Vitae},
  pdfsubject = {Curriculum Vitae},
  pdfpagemode = UseNone
}

% page size
\geometry{
  body={6.5in, 9.0in},
  left=1.0in,
  top=1.0in
}

% text formatting
\usepackage{color}
\definecolor{grey}{gray}{0.5}
\newcommand{\deemph}[1]{\textcolor{grey}{\footnotesize{#1}}}

% heading / footing
\usepackage{fancyheadings}
\pagestyle{fancy}
\renewcommand{\headrulewidth}{0pt}
\lhead{\deemph{Adrian M. Price-Whelan}}
\chead{\deemph{Curriculum Vitae}}
\rhead{\deemph{\thepage}}
\cfoot{\deemph{Last updated: \today}}

% Don't indent paragraphs.
\setlength\parindent{0em}

% Make lists without bullets and compact spacing
\renewenvironment{itemize}{
  \begin{list}{}{
    \setlength{\leftmargin}{1em}
    \setlength{\itemsep}{0.em}
    \setlength{\parskip}{0pt}
    \setlength{\parsep}{0.25em}
    \setlength{\itemindent}{-1em}
  }
}{
  \end{list}
}

% Change section font size and spacing
\usepackage{titlesec}
\titleformat{\section}{\normalfont\fontsize{14pt}{0}\bfseries}{\thesection}{}{}
\titleformat{\subsection}{\normalfont\fontsize{12pt}{0}\bfseries}{\thesubsection}{}{}
\titlespacing{\section}{0em}{-0.em}{0.25em}
\titlespacing{\subsection}{0.5em}{0.em}{0.25em}

% literature links (thanks @dfm)
\newcommand{\doi}[2]{\emph{\href{http://dx.doi.org/#1}{{#2}}}}
\newcommand{\ads}[2]{\href{http://adsabs.harvard.edu/abs/#1}{{#2}}}
\newcommand{\arxiv}[1]{{\href{http://arxiv.org/abs/#1}{arXiv:{#1}}}}

% Journal names
\newcommand{\aanda}{A\&A}
\newcommand{\aj}{AJ}
\newcommand{\apj}{ApJ}
\newcommand{\apjs}{ApJS}
\newcommand{\apjl}{ApJL}
\newcommand{\pasp}{PASP}
\newcommand{\mnras}{MNRAS}
\newcommand{\mnrasl}{MNRAS Letters}
\usepackage{multicol}

\begin{document}\thispagestyle{empty}\sloppy\sloppypar


\headersection{Adrian M. Price-Whelan}{Curriculum Vitae}


\begin{itemize}
  \item
    \textbf{Associate Research Scientist \& Leader of Galactic Dynamics Group} \\
    Center for Computational Astrophysics, Flatiron Institute, \\
    162 Fifth Ave., New York, NY 10010, USA
  \item \textbf{Assistant Director of Scientific Software, Simons Foundation}\\
    Math and Physical Sciences, Simons Foundation, \\
    160 Fifth Ave., New York, NY 10010, USA
  \item
    \href{mailto:adrianmpw@gmail.com}{\faEnvelope ~~ adrianmpw@gmail.com}
    ~~~
	\href{http://adrian.pw}{\faExternalLink* ~~ adrian.pw}
    ~~~
	\href{https://github.com/adrn}{\faGithub ~~ github.com/adrn}
    ~~~
    \href{\arxivurl}{\faFile ~~ arXiv}
\end{itemize}


% \section*{Research interests}
% \vspace{-1.4em}
% \begin{multicols}{3}
%     \begin{itemize}
%         \item dark matter
%         \item near-field cosmology
%         \item Milky Way
%         \item stellar streams
%         \item galactic dynamics
%         \item binary stars
%         \item star clusters
%         \item galactic archaeology
%         \item statistical methods
%     \end{itemize}
% \end{multicols}


\section*{Education and past positions}
	\begin{itemize}
        \item 2019--2021, Flatiron Research Fellow, Flatiron Institute
        \item 2016--2019, Lyman J. Spitzer, Jr. fellow, Princeton University
	\item Ph. D. 2016, Astronomy, Columbia University.
		{\it Advisor: K. V. Johnston}
	\item M.A. 2013, MPhil 2014, Astronomy, Columbia University.
		{\it Advisor: K. V. Johnston}
	\item Honors B.A. 2010, Physics, New York University.
		{\it Advisor: D. W. Hogg}
	\end{itemize}

% First author:
% https://ui.adsabs.harvard.edu/#search/fq=%7B!type%3Daqp%20v%3D%24fq_bibstem_facet%7D&fq_bibstem_facet=(bibstem_facet%3A%22ApJ%22%20OR%20bibstem_facet%3A%22ApJL%22%20OR%20bibstem_facet%3A%22JOSS%22%20OR%20bibstem_facet%3A%22MNRAS%22%20OR%20bibstem_facet%3A%22AJ%22%20OR%20bibstem_facet%3A%22arXiv%22%20OR%20bibstem_facet%3A%22PASP%22)&q=%20%20author%3A%22%5Eprice-whelan%2C%20a%22&sort=date%20desc%2C%20bibcode%20desc&p_=0

% Refereed:
% https://ui.adsabs.harvard.edu/#search/filter_property_fq_property=AND&filter_property_fq_property=property%3A%22refereed%22&fq=%7B!type%3Daqp%20v%3D%24fq_bibstem_facet%7D&fq=%7B!type%3Daqp%20v%3D%24fq_property%7D&fq_bibstem_facet=(bibstem_facet%3A%22NatAs%22%20OR%20bibstem_facet%3A%22Natur%22%20OR%20bibstem_facet%3A%22Galax%22%20OR%20bibstem_facet%3A%22PASP%22%20OR%20bibstem_facet%3A%22A%26A%22%20OR%20bibstem_facet%3A%22PASJ%22%20OR%20bibstem_facet%3A%22JOSS%22%20OR%20bibstem_facet%3A%22arXiv%22%20OR%20bibstem_facet%3A%22ApJS%22%20OR%20bibstem_facet%3A%22ApJL%22%20OR%20bibstem_facet%3A%22MNRAS%22%20OR%20bibstem_facet%3A%22AJ%22%20OR%20bibstem_facet%3A%22ApJ%22)&fq_property=(property%3A%22refereed%22)&q=%20%20author%3A%22price-whelan%2C%20a%22&sort=date%20desc%2C%20bibcode%20desc&p_=0

\section*{Publications}
\vspace{-1em}

\href{https://adrian.pw/cv/PriceWhelan-pubs.pdf}{{\it Full publication list} \faLink}
--- \href{\adsurl}{{\it ADS search} \faLink}

Refereed: 93 --- First author: 18 --- Citations: 25120 --- h-index: 39  (\textit{2022-07-26})

\subsection*{Highlighted and Recent Work}
    \begin{itemize}
        \item[{\color{deemph}\scriptsize15}]Garavito-Camargo, N.; Besla, G.; Laporte, C. F. P.; \textbf{Price-Whelan,~A.~M.}~\textit{et al.}, \doi{10.3847/1538-4357/ac0b44}{Quantifying the Impact of the Large Magellanic Cloud on the Structure of the Milky Way's Dark Matter Halo Using Basis Function Expansions}, \apj, \textbf{919}, 109, 2021 (\arxiv{2010.00816}) [\href{http://adsabs.harvard.edu/abs/2021ApJ...919..109G}{32 citations}]

\item[{\color{deemph}\scriptsize14}]\textbf{Price-Whelan,~A.~M.}; Hogg, D. W.; Johnston, K. V.; Ness, M. K.~\textit{et al.}, \doi{10.3847/1538-4357/abe1b7}{Orbital Torus Imaging: Using Element Abundances to Map Orbits and Mass in the Milky Way}, \apj, \textbf{910}, 17, 2021 (\arxiv{2012.00015}) [\href{http://adsabs.harvard.edu/abs/2021ApJ...910...17P}{7 citations}]

\item[{\color{deemph}\scriptsize13}]Shipp, N.; \textbf{Price-Whelan,~A.~M.}; Tavangar, K.; Mateu, C.~\textit{et al.}, \doi{10.3847/1538-3881/abbd3a}{Discovery of Extended Tidal Tails around the Globular Cluster Palomar 13}, \aj, \textbf{160}, 244, 2020 (\arxiv{2006.12501}) [\href{http://adsabs.harvard.edu/abs/2020AJ....160..244S}{13 citations}]

\item[{\color{deemph}\scriptsize12}]\textbf{Price-Whelan,~A.~M.}; Hogg, D. W.; Rix, H.; Beaton, R. L.~\textit{et al.}, \doi{10.3847/1538-4357/ab8acc}{Close Binary Companions to APOGEE DR16 Stars: 20,000 Binary-star Systems Across the Color-Magnitude Diagram}, \apj, \textbf{895}, 2, 2020 (\arxiv{2002.00014}) [\href{http://adsabs.harvard.edu/abs/2020ApJ...895....2P}{54 citations}]

\item[{\color{deemph}\scriptsize11}]\textbf{Price-Whelan,~A.~M.}; Nidever, D. L.; Choi, Y.; Schlafly, E. F.~\textit{et al.}, \doi{10.3847/1538-4357/ab4bdd}{Discovery of a Disrupting Open Cluster Far into the Milky Way Halo: A Recent Star Formation Event in the Leading Arm of the Magellanic Stream?}, \apj, \textbf{887}, 19, 2019 (\arxiv{1811.05991}) [\href{http://adsabs.harvard.edu/abs/2019ApJ...887...19P}{19 citations}]

\item[{\color{deemph}\scriptsize10}]Bonaca, A.; Hogg, D. W.; \textbf{Price-Whelan,~A.~M.}; Conroy, C., \doi{10.3847/1538-4357/ab2873}{The Spur and the Gap in GD-1: Dynamical Evidence for a Dark Substructure in the Milky Way Halo}, \apj, \textbf{880}, 38, 2019 (\arxiv{1811.03631}) [\href{http://adsabs.harvard.edu/abs/2019ApJ...880...38B}{114 citations}]

\item[{\color{deemph}\scriptsize9}]\textbf{Price-Whelan,~A.~M.}; Bonaca, A., \doi{10.3847/2041-8213/aad7b5}{Off the Beaten Path: Gaia Reveals GD-1 Stars outside of the Main Stream}, \apj, \textbf{863}, 2018 (\arxiv{1805.00425}) [\href{http://adsabs.harvard.edu/abs/2018ApJ...863L..20P}{79 citations}]

\item[{\color{deemph}\scriptsize8}]\textbf{Price-Whelan,~A.~M.}; Hogg, D. W.; Rix, H.; De Lee, N.~\textit{et al.}, \doi{10.3847/1538-3881/aac387}{Binary Companions of Evolved Stars in APOGEE DR14: Search Method and Catalog of {\ensuremath{\sim}}5000 Companions}, \aj, \textbf{156}, 18, 2018 (\arxiv{1804.04662}) [\href{http://adsabs.harvard.edu/abs/2018AJ....156...18P}{431 citations}]

\item[{\color{deemph}\scriptsize7}]Oh, S.; \textbf{Price-Whelan,~A.~M.}; Brewer, J. M.; Hogg, D. W.~\textit{et al.}, \doi{10.3847/1538-4357/aaab4d}{Kronos and Krios: Evidence for Accretion of a Massive, Rocky Planetary System in a Comoving Pair of Solar-type Stars}, \apj, \textbf{854}, 138, 2018 (\arxiv{1709.05344}) [\href{http://adsabs.harvard.edu/abs/2018ApJ...854..138O}{45 citations}]

\item[{\color{deemph}\scriptsize6}]\textbf{Price-Whelan,~A.~M.}, \doi{10.21105/joss.00388}{Gala: A Python package for galactic dynamics}, JOSS, \textbf{2}, 388, 2017 [\href{http://adsabs.harvard.edu/abs/2017JOSS....2..388P}{99 citations}]

\item[{\color{deemph}\scriptsize5}]Oh, S.; \textbf{Price-Whelan,~A.~M.}; Hogg, D. W.; Morton, T. D.~\textit{et al.}, \doi{10.3847/1538-3881/aa6ffd}{Comoving Stars in Gaia DR1: An Abundance of Very Wide Separation Comoving Pairs}, \aj, \textbf{153}, 257, 2017 (\arxiv{1612.02440}) [\href{http://adsabs.harvard.edu/abs/2017AJ....153..257O}{117 citations}]

\item[{\color{deemph}\scriptsize4}]\textbf{Price-Whelan,~A.~M.}; Hogg, D. W.; Foreman-Mackey, D.; Rix, H., \doi{10.3847/1538-4357/aa5e50}{The Joker: A Custom Monte Carlo Sampler for Binary-star and Exoplanet Radial Velocity Data}, \apj, \textbf{837}, 20, 2017 (\arxiv{1610.07602}) [\href{http://adsabs.harvard.edu/abs/2017ApJ...837...20P}{70 citations}]

\item[{\color{deemph}\scriptsize3}]\textbf{Price-Whelan,~A.~M.}; Johnston, K. V.; Valluri, M.; Pearson, S.~\textit{et al.}, \doi{10.1093/mnras/stv2383}{Chaotic dispersal of tidal debris}, \mnras, \textbf{455}, 1079, 2016 (\arxiv{1507.08662}) [\href{http://adsabs.harvard.edu/abs/2016MNRAS.455.1079P}{48 citations}]

\item[{\color{deemph}\scriptsize2}]\textbf{Price-Whelan,~A.~M.}; Johnston, K. V.; Sheffield, A. A.; Laporte, C. F. P.~\textit{et al.}, \doi{10.1093/mnras/stv1324}{A reinterpretation of the Triangulum-Andromeda stellar clouds: a population of halo stars kicked out of the Galactic disc}, \mnras, \textbf{452}, 676, 2015 (\arxiv{1503.08780}) [\href{http://adsabs.harvard.edu/abs/2015MNRAS.452..676P}{81 citations}]

\item[{\color{deemph}\scriptsize1}]\textbf{Price-Whelan,~A.~M.}; Hogg, D. W.; Johnston, K. V.; Hendel, D., \doi{10.1088/0004-637X/794/1/4}{Inferring the Gravitational Potential of the Milky Way with a Few Precisely Measured Stars}, \apj, \textbf{794}, 4, 2014 (\arxiv{1405.6721}) [\href{http://adsabs.harvard.edu/abs/2014ApJ...794....4P}{45 citations}]
    \end{itemize}


% \subsection*{Refereed}
%     \begin{itemize}
%         \item[{\color{deemph}\scriptsize51}]Erkal, D.~\textit{et al.}~(incl. \textbf{APW}), \doi{10.1093/mnras/stz1371}{The total mass of the Large Magellanic Cloud from its perturbation on the Orphan stream}, \mnras, \textbf{487}, 2685, 2019 (\arxiv{1812.08192}) [\href{http://adsabs.harvard.edu/abs/2019MNRAS.487.2685E}{15 citations}]

\item[{\color{deemph}\scriptsize50}]Bonaca, A.; Hogg, D. W.; \textbf{Price-Whelan,~A.~M.}; Conroy, C., \doi{10.3847/1538-4357/ab2873}{The Spur and the Gap in GD-1: Dynamical Evidence for a Dark Substructure in the Milky Way Halo}, \apj, \textbf{880}, 38, 2019 (\arxiv{1811.03631}) [\href{http://adsabs.harvard.edu/abs/2019ApJ...880...38B}{17 citations}]

\item[{\color{deemph}\scriptsize49}]Rasskazov, A.~\textit{et al.}~(incl. \textbf{APW}), \doi{10.3847/1538-4357/ab1c5d}{Hypervelocity Stars from a Supermassive Black Hole--Intermediate-mass Black Hole Binary}, \apj, \textbf{878}, 17, 2019 (\arxiv{1810.12354}) [\href{http://adsabs.harvard.edu/abs/2019ApJ...878...17R}{2 citations}]

\item[{\color{deemph}\scriptsize48}]Koposov, S. E.~\textit{et al.}~(incl. \textbf{APW}), \doi{10.1093/mnras/stz457}{Piercing the Milky Way: an all-sky view of the Orphan Stream}, \mnras, \textbf{485}, 4726, 2019 (\arxiv{1812.08172}) [\href{http://adsabs.harvard.edu/abs/2019MNRAS.485.4726K}{11 citations}]

\item[{\color{deemph}\scriptsize47}]\textbf{Price-Whelan,~A.~M.}; Goodman, J., \doi{10.3847/1538-4357/aae264}{Binary Companions of Evolved Stars in APOGEE DR14: Orbital Circularization}, \apj, \textbf{867}, 5, 2018 (\arxiv{1804.06841}) [\href{http://adsabs.harvard.edu/abs/2018ApJ...867....5P}{4 citations}]

\item[{\color{deemph}\scriptsize46}]De Rosa, G.~\textit{et al.}~(incl. \textbf{APW}), \doi{10.3847/1538-4357/aadd11}{Velocity-resolved Reverberation Mapping of Five Bright Seyfert 1 Galaxies}, \apj, \textbf{866}, 133, 2018 (\arxiv{1807.04784}) [\href{http://adsabs.harvard.edu/abs/2018ApJ...866..133D}{4 citations}]

\item[{\color{deemph}\scriptsize45}]Kado-Fong, E.; Greene, J. E.; Hendel, D.; \textbf{Price-Whelan,~A.~M.}~\textit{et al.}, \doi{10.3847/1538-4357/aae0f0}{Tidal Features at 0.05 {\&}lt; z {\&}lt; 0.45 in the Hyper Suprime-Cam Subaru Strategic Program: Properties and Formation Channels}, \apj, \textbf{866}, 103, 2018 (\arxiv{1805.05970}) [\href{http://adsabs.harvard.edu/abs/2018ApJ...866..103K}{8 citations}]

\item[{\color{deemph}\scriptsize44}]Anderson, L.; Hogg, D. W.; Leistedt, B.; \textbf{Price-Whelan,~A.~M.}~\textit{et al.}, \doi{10.3847/1538-3881/aad7bf}{Improving Gaia Parallax Precision with a Data-driven Model of Stars}, \aj, \textbf{156}, 145, 2018 (\arxiv{1706.05055}) [\href{http://adsabs.harvard.edu/abs/2018AJ....156..145A}{15 citations}]

\item[{\color{deemph}\scriptsize43}]Astropy Collaboration; \textbf{Price-Whelan,~A.~M.}; Sip{\'{o}}cz, B. M.; G{\"u}nther, H. M.~\textit{et al.}, \doi{10.3847/1538-3881/aabc4f}{The Astropy Project: Building an Open-science Project and Status of the v2.0 Core Package}, \aj, \textbf{156}, 123, 2018 (\arxiv{1801.02634}) [\href{http://adsabs.harvard.edu/abs/2018AJ....156..123A}{527 citations}]

\item[{\color{deemph}\scriptsize42}]Hendel, D.~\textit{et al.}~(incl. \textbf{APW}), \doi{10.1093/mnras/sty1455}{SMHASH: anatomy of the Orphan Stream using RR Lyrae stars}, \mnras, \textbf{479}, 570, 2018 (\arxiv{1711.04663}) [\href{http://adsabs.harvard.edu/abs/2018MNRAS.479..570H}{10 citations}]

\item[{\color{deemph}\scriptsize41}]\textbf{Price-Whelan,~A.~M.}; Bonaca, A., \doi{10.3847/2041-8213/aad7b5}{Off the Beaten Path: Gaia Reveals GD-1 Stars outside of the Main Stream}, \apj, \textbf{863}, 2018 (\arxiv{1805.00425}) [\href{http://adsabs.harvard.edu/abs/2018ApJ...863L..20P}{30 citations}]

\item[{\color{deemph}\scriptsize40}]\textbf{Price-Whelan,~A.~M.}; Hogg, D. W.; Rix, H.; De Lee, N.~\textit{et al.}, \doi{10.3847/1538-3881/aac387}{Binary Companions of Evolved Stars in APOGEE DR14: Search Method and Catalog of {\ensuremath{\sim}}5000 Companions}, \aj, \textbf{156}, 18, 2018 (\arxiv{1804.04662}) [\href{http://adsabs.harvard.edu/abs/2018AJ....156...18P}{50 citations}]

\item[{\color{deemph}\scriptsize39}]Hayes, C. R.~\textit{et al.}~(incl. \textbf{APW}), \doi{10.3847/2041-8213/aac38c}{Disk-like Chemistry of the Triangulum-Andromeda Overdensity as Seen by APOGEE}, \apj, \textbf{859}, 2018 (\arxiv{1805.03706}) [\href{http://adsabs.harvard.edu/abs/2018ApJ...859L...8H}{4 citations}]

\item[{\color{deemph}\scriptsize38}]Bergemann, M.~\textit{et al.}~(incl. \textbf{APW}), \doi{10.1038/nature25490}{Two chemically similar stellar overdensities on opposite sides of the plane of the Galactic disk}, \nature, \textbf{555}, 334, 2018 (\arxiv{1803.00563}) [\href{http://adsabs.harvard.edu/abs/2018Natur.555..334B}{22 citations}]

\item[{\color{deemph}\scriptsize37}]Morris, B. M.~\textit{et al.}~(incl. \textbf{APW}), \doi{10.3847/1538-3881/aaa47e}{astroplan: An Open Source Observation Planning Package in Python}, \aj, \textbf{155}, 128, 2018 (\arxiv{1712.09631}) [\href{http://adsabs.harvard.edu/abs/2018AJ....155..128M}{15 citations}]

\item[{\color{deemph}\scriptsize36}]Oh, S.; \textbf{Price-Whelan,~A.~M.}; Brewer, J. M.; Hogg, D. W.~\textit{et al.}, \doi{10.3847/1538-4357/aaab4d}{Kronos and Krios: Evidence for Accretion of a Massive, Rocky Planetary System in a Comoving Pair of Solar-type Stars}, \apj, \textbf{854}, 138, 2018 (\arxiv{1709.05344}) [\href{http://adsabs.harvard.edu/abs/2018ApJ...854..138O}{18 citations}]

\item[{\color{deemph}\scriptsize35}]Sheffield, A. A.; \textbf{Price-Whelan,~A.~M.}; Tzanidakis, A.; Johnston, K. V.~\textit{et al.}, \doi{10.3847/1538-4357/aaa4b6}{A Disk Origin for the Monoceros Ring and A13 Stellar Overdensities}, \apj, \textbf{854}, 47, 2018 (\arxiv{1801.01171}) [\href{http://adsabs.harvard.edu/abs/2018ApJ...854...47S}{11 citations}]

\item[{\color{deemph}\scriptsize34}]Greco, J. P.; Greene, J. E.; \textbf{Price-Whelan,~A.~M.}; Leauthaud, A.~\textit{et al.}, \doi{10.1093/pasj/psx051}{Sumo Puff: Tidal debris or disturbed ultra-diffuse galaxy?}, \pasj, \textbf{70}, 2018 (\arxiv{1704.06681}) [\href{http://adsabs.harvard.edu/abs/2018PASJ...70S..19G}{11 citations}]

\item[{\color{deemph}\scriptsize33}]Goulding, A. D.~\textit{et al.}~(incl. \textbf{APW}), \doi{10.1093/pasj/psx135}{Galaxy interactions trigger rapid black hole growth: An unprecedented view from the Hyper Suprime-Cam survey}, \pasj, \textbf{70}, 2018 (\arxiv{1706.07436}) [\href{http://adsabs.harvard.edu/abs/2018PASJ...70S..37G}{34 citations}]

\item[{\color{deemph}\scriptsize32}]\textbf{Price-Whelan,~A.~M.}, \doi{10.21105/joss.00388}{Gala: A Python package for galactic dynamics}, JOSS, \textbf{2}, 388, 2017 [\href{http://adsabs.harvard.edu/abs/2017JOSS....2..388P}{22 citations}]

\item[{\color{deemph}\scriptsize31}]Alam, S.~\textit{et al.}~(incl. \textbf{APW}), \doi{10.1093/mnras/stx721}{The clustering of galaxies in the completed SDSS-III Baryon Oscillation Spectroscopic Survey: cosmological analysis of the DR12 galaxy sample}, \mnras, \textbf{470}, 2617, 2017 (\arxiv{1607.03155}) [\href{http://adsabs.harvard.edu/abs/2017MNRAS.470.2617A}{588 citations}]

\item[{\color{deemph}\scriptsize30}]\textbf{Price-Whelan,~A.~M.}; Foreman-Mackey, D., \doi{10.21105/joss.00357}{schwimmbad: A uniform interface to parallel processing pools in Python}, JOSS, \textbf{2}, 357, 2017 [\href{http://adsabs.harvard.edu/abs/2017JOSS....2..357P}{5 citations}]

\item[{\color{deemph}\scriptsize29}]Pearson, S.; \textbf{Price-Whelan,~A.~M.}; Johnston, K. V., \doi{10.1038/s41550-017-0220-3}{Gaps and length asymmetry in the stellar stream Palomar 5 as effects of Galactic bar rotation}, \natureast, \textbf{1}, 633, 2017 (\arxiv{1703.04627}) [\href{http://adsabs.harvard.edu/abs/2017NatAs...1..633P}{28 citations}]

\item[{\color{deemph}\scriptsize28}]Johnston, K. V.; \textbf{Price-Whelan,~A.~M.}; Bergemann, M.; Laporte, C.~\textit{et al.}, \doi{10.3390/galaxies5030044}{Disk Heating, Galactoseismology, and the Formation of Stellar Halos}, MDPI: galaxies, \textbf{5}, 44, 2017 (\arxiv{1709.00491}) [\href{http://adsabs.harvard.edu/abs/2017Galax...5...44J}{5 citations}]

\item[{\color{deemph}\scriptsize27}]Li, T. S.~\textit{et al.}~(incl. \textbf{APW}), \doi{10.3847/1538-4357/aa7a0d}{Exploring Halo Substructure with Giant Stars. XV. Discovery of a Connection between the Monoceros Ring and the Triangulum-Andromeda Overdensity?}, \apj, \textbf{844}, 74, 2017 (\arxiv{1703.05384}) [\href{http://adsabs.harvard.edu/abs/2017ApJ...844...74L}{14 citations}]

\item[{\color{deemph}\scriptsize26}]Oh, S.; \textbf{Price-Whelan,~A.~M.}; Hogg, D. W.; Morton, T. D.~\textit{et al.}, \doi{10.3847/1538-3881/aa6ffd}{Comoving Stars in Gaia DR1: An Abundance of Very Wide Separation Comoving Pairs}, \aj, \textbf{153}, 257, 2017 (\arxiv{1612.02440}) [\href{http://adsabs.harvard.edu/abs/2017AJ....153..257O}{48 citations}]

\item[{\color{deemph}\scriptsize25}]Sesar, B.; Fouesneau, M.; \textbf{Price-Whelan,~A.~M.}; Bailer-Jones, C. A. L.~\textit{et al.}, \doi{10.3847/1538-4357/aa643b}{A Probabilistic Approach to Fitting Period--luminosity Relations and Validating Gaia Parallaxes}, \apj, \textbf{838}, 107, 2017 (\arxiv{1611.07035}) [\href{http://adsabs.harvard.edu/abs/2017ApJ...838..107S}{30 citations}]

\item[{\color{deemph}\scriptsize24}]\textbf{Price-Whelan,~A.~M.}; Hogg, D. W.; Foreman-Mackey, D.; Rix, H., \doi{10.3847/1538-4357/aa5e50}{The Joker: A Custom Monte Carlo Sampler for Binary-star and Exoplanet Radial Velocity Data}, \apj, \textbf{837}, 20, 2017 (\arxiv{1610.07602}) [\href{http://adsabs.harvard.edu/abs/2017ApJ...837...20P}{19 citations}]

\item[{\color{deemph}\scriptsize23}]Charisi, M.; Bartos, I.; Haiman, Z.; \textbf{Price-Whelan,~A.~M.}~\textit{et al.}, \doi{10.1093/mnras/stw1838}{A population of short-period variable quasars from PTF as supermassive black hole binary candidates}, \mnras, \textbf{463}, 2145, 2016 (\arxiv{1604.01020}) [\href{http://adsabs.harvard.edu/abs/2016MNRAS.463.2145C}{64 citations}]

\item[{\color{deemph}\scriptsize22}]\textbf{Price-Whelan,~A.~M.}; Sesar, B.; Johnston, K. V.; Rix, H., \doi{10.3847/0004-637X/824/2/104}{Spending Too Much Time at the Galactic Bar: Chaotic Fanning of the Ophiuchus Stream}, \apj, \textbf{824}, 104, 2016 (\arxiv{1601.06790}) [\href{http://adsabs.harvard.edu/abs/2016ApJ...824..104P}{18 citations}]

\item[{\color{deemph}\scriptsize21}]Sesar, B.; \textbf{Price-Whelan,~A.~M.}; Cohen, J. G.; Rix, H.~\textit{et al.}, \doi{10.3847/2041-8205/816/1/L4}{Evidence of Fanning in the Ophiuchus Stream}, \apj, \textbf{816}, 2016 (\arxiv{1512.00469}) [\href{http://adsabs.harvard.edu/abs/2016ApJ...816L...4S}{4 citations}]

\item[{\color{deemph}\scriptsize20}]\textbf{Price-Whelan,~A.~M.}; Johnston, K. V.; Valluri, M.; Pearson, S.~\textit{et al.}, \doi{10.1093/mnras/stv2383}{Chaotic dispersal of tidal debris}, \mnras, \textbf{455}, 1079, 2016 (\arxiv{1507.08662}) [\href{http://adsabs.harvard.edu/abs/2016MNRAS.455.1079P}{25 citations}]

\item[{\color{deemph}\scriptsize19}]Charisi, M.; Bartos, I.; Haiman, Z.; \textbf{Price-Whelan,~A.~M.}~\textit{et al.}, \doi{10.1093/mnrasl/slv111}{Multiple periods in the variability of the supermassive black hole binary candidate quasar PG1302-102?}, \mnras, \textbf{454}, 2015 (\arxiv{1502.03113}) [\href{http://adsabs.harvard.edu/abs/2015MNRAS.454L..21C}{16 citations}]

\item[{\color{deemph}\scriptsize18}]\textbf{Price-Whelan,~A.~M.}; Johnston, K. V.; Sheffield, A. A.; Laporte, C. F. P.~\textit{et al.}, \doi{10.1093/mnras/stv1324}{A reinterpretation of the Triangulum-Andromeda stellar clouds: a population of halo stars kicked out of the Galactic disc}, \mnras, \textbf{452}, 676, 2015 (\arxiv{1503.08780}) [\href{http://adsabs.harvard.edu/abs/2015MNRAS.452..676P}{46 citations}]

\item[{\color{deemph}\scriptsize17}]Sesar, B.~\textit{et al.}~(incl. \textbf{APW}), \doi{10.1088/0004-637X/809/1/59}{The Nature and Orbit of the Ophiuchus Stream}, \apj, \textbf{809}, 59, 2015 (\arxiv{1501.00581}) [\href{http://adsabs.harvard.edu/abs/2015ApJ...809...59S}{22 citations}]

\item[{\color{deemph}\scriptsize16}]Alam, S.~\textit{et al.}~(incl. \textbf{APW}), \doi{10.1088/0067-0049/219/1/12}{The Eleventh and Twelfth Data Releases of the Sloan Digital Sky Survey: Final Data from SDSS-III}, \apjs, \textbf{219}, 12, 2015 (\arxiv{1501.00963}) [\href{http://adsabs.harvard.edu/abs/2015ApJS..219...12A}{1099 citations}]

\item[{\color{deemph}\scriptsize15}]Pearson, S.; K{\"u}pper, A. H. W.; Johnston, K. V.; \textbf{Price-Whelan,~A.~M.}, \doi{10.1088/0004-637X/799/1/28}{Tidal Stream Morphology as an Indicator of Dark Matter Halo Geometry: The Case of Palomar 5}, \apj, \textbf{799}, 28, 2015 (\arxiv{1410.3477}) [\href{http://adsabs.harvard.edu/abs/2015ApJ...799...28P}{42 citations}]

\item[{\color{deemph}\scriptsize14}]Andrews, J. J.; \textbf{Price-Whelan,~A.~M.}; Ag{\"u}eros, M. A., \doi{10.1088/2041-8205/797/2/L32}{The Mass Distribution of Companions to Low-mass White Dwarfs}, \apj, \textbf{797}, 2014 (\arxiv{1412.0114}) [\href{http://adsabs.harvard.edu/abs/2014ApJ...797L..32A}{15 citations}]

\item[{\color{deemph}\scriptsize13}]\textbf{Price-Whelan,~A.~M.}; Hogg, D. W.; Johnston, K. V.; Hendel, D., \doi{10.1088/0004-637X/794/1/4}{Inferring the Gravitational Potential of the Milky Way with a Few Precisely Measured Stars}, \apj, \textbf{794}, 4, 2014 (\arxiv{1405.6721}) [\href{http://adsabs.harvard.edu/abs/2014ApJ...794....4P}{32 citations}]

\item[{\color{deemph}\scriptsize12}]Anderson, L.~\textit{et al.}~(incl. \textbf{APW}), \doi{10.1093/mnras/stu523}{The clustering of galaxies in the SDSS-III Baryon Oscillation Spectroscopic Survey: baryon acoustic oscillations in the Data Releases 10 and 11 Galaxy samples}, \mnras, \textbf{441}, 24, 2014 (\arxiv{1312.4877}) [\href{http://adsabs.harvard.edu/abs/2014MNRAS.441...24A}{869 citations}]

\item[{\color{deemph}\scriptsize11}]Ahn, C. P.~\textit{et al.}~(incl. \textbf{APW}), \doi{10.1088/0067-0049/211/2/17}{The Tenth Data Release of the Sloan Digital Sky Survey: First Spectroscopic Data from the SDSS-III Apache Point Observatory Galactic Evolution Experiment}, \apjs, \textbf{211}, 17, 2014 (\arxiv{1307.7735}) [\href{http://adsabs.harvard.edu/abs/2014ApJS..211...17A}{720 citations}]

\item[{\color{deemph}\scriptsize10}]\textbf{Price-Whelan,~A.~M.}; Ag{\"u}eros, M. A.; Fournier, A. P.; Street, R.~\textit{et al.}, \doi{10.1088/0004-637X/781/1/35}{Statistical Searches for Microlensing Events in Large, Non-uniformly Sampled Time-Domain Surveys: A Test Using Palomar Transient Factory Data}, \apj, \textbf{781}, 35, 2014 (\arxiv{1311.3683}) [\href{http://adsabs.harvard.edu/abs/2014ApJ...781...35P}{7 citations}]

\item[{\color{deemph}\scriptsize9}]\textbf{Price-Whelan,~A.~M.}; Johnston, K. V., \doi{10.1088/2041-8205/778/1/L12}{Spitzer, Gaia, and the Potential of the Milky Way}, \apj, \textbf{778}, 2013 (\arxiv{1308.2670}) [\href{http://adsabs.harvard.edu/abs/2013ApJ...778L..12P}{26 citations}]

\item[{\color{deemph}\scriptsize8}]Astropy Collaboration~\textit{et al.}~(incl. \textbf{APW}), \doi{10.1051/0004-6361/201322068}{Astropy: A community Python package for astronomy}, \aanda, \textbf{558}, 2013 (\arxiv{1307.6212}) [\href{http://adsabs.harvard.edu/abs/2013A&A...558A..33A}{2153 citations}]

\item[{\color{deemph}\scriptsize7}]Dawson, K. S.~\textit{et al.}~(incl. \textbf{APW}), \doi{10.1088/0004-6256/145/1/10}{The Baryon Oscillation Spectroscopic Survey of SDSS-III}, \aj, \textbf{145}, 10, 2013 (\arxiv{1208.0022}) [\href{http://adsabs.harvard.edu/abs/2013AJ....145...10D}{1044 citations}]

\item[{\color{deemph}\scriptsize6}]Ahn, C. P.~\textit{et al.}~(incl. \textbf{APW}), \doi{10.1088/0067-0049/203/2/21}{The Ninth Data Release of the Sloan Digital Sky Survey: First Spectroscopic Data from the SDSS-III Baryon Oscillation Spectroscopic Survey}, \apjs, \textbf{203}, 21, 2012 (\arxiv{1207.7137}) [\href{http://adsabs.harvard.edu/abs/2012ApJS..203...21A}{947 citations}]

\item[{\color{deemph}\scriptsize5}]Eisenstein, D. J.~\textit{et al.}~(incl. \textbf{APW}), \doi{10.1088/0004-6256/142/3/72}{SDSS-III: Massive Spectroscopic Surveys of the Distant Universe, the Milky Way, and Extra-Solar Planetary Systems}, \aj, \textbf{142}, 72, 2011 (\arxiv{1101.1529}) [\href{http://adsabs.harvard.edu/abs/2011AJ....142...72E}{1272 citations}]

\item[{\color{deemph}\scriptsize4}]Aihara, H.~\textit{et al.}~(incl. \textbf{APW}), \doi{10.1088/0067-0049/195/2/26}{Erratum: ''The Eighth Data Release of the Sloan Digital Sky Survey: First Data from SDSS-III'' <A href=''/abs/2011ApJS..193...29A''>(2011, ApJS, 193, 29)</A>}, \apjs, \textbf{195}, 26, 2011 [\href{http://adsabs.harvard.edu/abs/2011ApJS..195...26A}{45 citations}]

\item[{\color{deemph}\scriptsize3}]Blanton, M. R.~\textit{et al.}~(incl. \textbf{APW}), \doi{10.1088/0004-6256/142/1/31}{Improved Background Subtraction for the Sloan Digital Sky Survey Images}, \aj, \textbf{142}, 31, 2011 (\arxiv{1105.1960}) [\href{http://adsabs.harvard.edu/abs/2011AJ....142...31B}{180 citations}]

\item[{\color{deemph}\scriptsize2}]Aihara, H.~\textit{et al.}~(incl. \textbf{APW}), \doi{10.1088/0067-0049/193/2/29}{The Eighth Data Release of the Sloan Digital Sky Survey: First Data from SDSS-III}, \apjs, \textbf{193}, 29, 2011 (\arxiv{1101.1559}) [\href{http://adsabs.harvard.edu/abs/2011ApJS..193...29A}{994 citations}]

\item[{\color{deemph}\scriptsize1}]\textbf{Price-Whelan,~A.~M.}; Hogg, D. W., \doi{10.1086/651009}{What Bandwidth Do I Need for My Image?}, \pasp, \textbf{122}, 207, 2010 (\arxiv{0910.2375}) [\href{http://adsabs.harvard.edu/abs/2010PASP..122..207P}{4 citations}]
%     \end{itemize}

% \subsection*{Preprints \& other}
%     \begin{itemize}
%         \item[{\color{deemph}\scriptsize17}]Aganze, C.~\textit{et al.}~(incl. \textbf{APW}), \doi{10.48550/arXiv.2305.12045}{Prospects for Detecting Gaps in Globular Cluster Stellar Streams in External Galaxies with the Nancy Grace Roman Space Telescope}, 2023 (\arxiv{2305.12045})

\item[{\color{deemph}\scriptsize16}]Castro-Ginard, A.~\textit{et al.}~(incl. \textbf{APW}), \doi{10.48550/arXiv.2303.17738}{Estimating the selection function of Gaia DR3 sub-samples}, 2023 (\arxiv{2303.17738})

\item[{\color{deemph}\scriptsize15}]Darragh-Ford, E.; Hunt, J. A. S.; \textbf{Price-Whelan,~A.~M.}; Johnston, K. V., \doi{10.48550/arXiv.2302.09086}{$\texttt{\{}ESCARGOT{\}}$: Mapping Vertical Phase Spiral Characteristics Throughout the Real and Simulated Milky Way}, 2023 (\arxiv{2302.09086}) [\href{http://adsabs.harvard.edu/abs/2023arXiv230209086D}{2 citations}]

\item[{\color{deemph}\scriptsize14}]Yavetz, T. D.; Johnston, K. V.; Pearson, S.; \textbf{Price-Whelan,~A.~M.}~\textit{et al.}, \doi{10.48550/arXiv.2212.11006}{Stream Fanning and Bifurcations: Observable Signatures of Resonances in Stellar Stream Morphology}, 2022 (\arxiv{2212.11006})

\item[{\color{deemph}\scriptsize13}]Breivik, K.~\textit{et al.}~(incl. \textbf{APW}), \doi{10.48550/arXiv.2208.02781}{From Data to Software to Science with the Rubin Observatory LSST}, 2022 (\arxiv{2208.02781}) [\href{http://adsabs.harvard.edu/abs/2022arXiv220802781B}{2 citations}]

\item[{\color{deemph}\scriptsize12}]Chance, Q.~\textit{et al.}~(incl. \textbf{APW}), \doi{10.48550/arXiv.2206.11275}{paired: A Statistical Framework for Determining Stellar Binarity with Gaia RVs. I. Planet Hosting Binaries}, 2022 (\arxiv{2206.11275})

\item[{\color{deemph}\scriptsize11}]Lucey, M.~\textit{et al.}~(incl. \textbf{APW}), \doi{10.48550/arXiv.2206.08299}{Over 2.7 Million Carbon-Enhanced Metal-Poor stars from BP/RP Spectra in $Gaia$ DR3}, 2022 (\arxiv{2206.08299}) [\href{http://adsabs.harvard.edu/abs/2022arXiv220608299L}{5 citations}]

\item[{\color{deemph}\scriptsize10}]Bechtol, K.~\textit{et al.}~(incl. \textbf{APW}), \doi{10.48550/arXiv.2203.07354}{Snowmass2021 Cosmic Frontier White Paper: Dark Matter Physics from Halo Measurements}, 2022 (\arxiv{2203.07354}) [\href{http://adsabs.harvard.edu/abs/2022arXiv220307354B}{24 citations}]

\item[{\color{deemph}\scriptsize9}]Katz, D. S.~\textit{et al.}~(incl. \textbf{APW}), \doi{10.48550/arXiv.2010.05102}{Software Sustainability {\&}amp; High Energy Physics}, 2020 (\arxiv{2010.05102}) [\href{http://adsabs.harvard.edu/abs/2020arXiv201005102K}{2 citations}]

\item[{\color{deemph}\scriptsize8}]Oladosu, A.~\textit{et al.}~(incl. \textbf{APW}), \doi{10.48550/arXiv.2007.04459}{Meta-Learning for One-Class Classification with Few Examples using Order-Equivariant Network}, 2020 (\arxiv{2007.04459}) [\href{http://adsabs.harvard.edu/abs/2020arXiv200704459O}{4 citations}]

\item[{\color{deemph}\scriptsize7}]Hogg, D. W.; \textbf{Price-Whelan,~A.~M.}; Leistedt, B., \doi{10.48550/arXiv.2005.14199}{Data Analysis Recipes: Products of multivariate Gaussians in Bayesian inferences}, 2020 (\arxiv{2005.14199}) [\href{http://adsabs.harvard.edu/abs/2020arXiv200514199H}{3 citations}]

\item[{\color{deemph}\scriptsize6}]Ness, M.~\textit{et al.}~(incl. \textbf{APW}), \doi{10.48550/arXiv.1907.05422}{In Pursuit of Galactic Archaeology: Astro2020 Science White Paper}, 2019 (\arxiv{1907.05422})

\item[{\color{deemph}\scriptsize5}]Buckley, M. R.; Hogg, D. W.; \textbf{Price-Whelan,~A.~M.}, \doi{10.48550/arXiv.1907.00987}{Applying Liouville's Theorem to Gaia Data}, 2019 (\arxiv{1907.00987}) [\href{http://adsabs.harvard.edu/abs/2019arXiv190700987B}{3 citations}]

\item[{\color{deemph}\scriptsize4}]The MSE Science Team~\textit{et al.}~(incl. \textbf{APW}), \doi{10.48550/arXiv.1904.04907}{The Detailed Science Case for the Maunakea Spectroscopic Explorer, 2019 edition}, 2019 (\arxiv{1904.04907}) [\href{http://adsabs.harvard.edu/abs/2019arXiv190404907T}{58 citations}]

\item[{\color{deemph}\scriptsize3}]Breivik, K.; \textbf{Price-Whelan,~A.~M.}; D'Orazio, D. J.; Hogg, D. W.~\textit{et al.}, \doi{10.48550/arXiv.1903.05094}{Stellar multiplicity: an interdisciplinary nexus}, 2019 (\arxiv{1903.05094}) [\href{http://adsabs.harvard.edu/abs/2019arXiv190305094B}{2 citations}]

\item[{\color{deemph}\scriptsize2}]Bergemann, M.~\textit{et al.}~(incl. \textbf{APW}), \doi{10.48550/arXiv.1903.03157}{Stellar Astrophysics and Exoplanet Science with the Maunakea Spectroscopic Explorer (MSE)}, 2019 (\arxiv{1903.03157})

\item[{\color{deemph}\scriptsize1}]\textbf{Price-Whelan,~A.~M.}; Oh, S.; Spergel, D. N., \doi{10.48550/arXiv.1709.03532}{Spectroscopic confirmation of very-wide stellar binaries and large-separation comoving pairs from Gaia DR1}, 2017 (\arxiv{1709.03532}) [\href{http://adsabs.harvard.edu/abs/2017arXiv170903532P}{16 citations}]
%     \end{itemize}

\section*{Grants and observing}

    \begin{itemize}
    \item {\it Cold Dark Matter and the GD-1 Stellar Stream}, Hubble Space Telescope, Cycle 27, 2019
    \item {\it Spectroscopic follow-up of a young cluster near the Leading Arm of the Magellanic System}, Clay Telescope, MIKE, 2019
    \item {\it Three-dimensional kinematics of the GD-1 stellar stream}, MMT 6.5m, 2018 %(Co-I, 2018)
    \item {\it Comoving stars in Gaia DR1}, Hiltner Telescope, MDM, 2017 %(PI, 2017)
    \item {\it TRACSSS-2: Tracing More Cold Stellar Streams with Spitzer}, Spitzer mission, Cycle 13, 2016 %(Co-I, 2016-2017)
	\item {\it The Triangulum-Andromeda stellar clouds: a population of halo stars kicked out of the Galactic disk?}, Hiltner Telescope, MDM, 2015 %(PI, 2015)
	\item {\it Spitzer Merger History and Shape of the Galactic Halo}, Spitzer mission, Cycle 10, 2014 %(Co-I, 2014-2015)
	\item {\it Gaia, Spitzer, and the potential of the Milky Way}, NASA theory grant, 2014--2016 %(Co-I, 2014-2016)
	\item Sigma Xi Grants in Aid of Research, 2013 %(PI, 2013-2014)
	\item {\it Probing the Milky Way's dark matter halo with RR Lyraes}, Hiltner Telescope, MDM, 2013 %(PI, 2013)
	\end{itemize}

\section*{Honors and awards}

    \begin{itemize}
    \item Rising Star in Astronomy, Astronomy Magazine (November 2022)
    \item Blavatnik Regional Awards, Winner in Physical Sciences and Engineering (2020)
    \item Group Achievement Award from the Royal Astronomical Society (Astropy Project; 2020)
    \item Dr. Pliny A. and Margaret H. Price Prize in Cosmology and AstroParticle Physics (2015)
	\item NSF Graduate Research Fellowship (2012--2016)
	\item Survey architect, SDSS-III (2011--2014)
	% \item Dean's List, New York University (2007--2010)
	\item Phi Beta Kappa, Beta of New York (2010--2016)
	\item Summa cum laude, New York University (2010)
	\item Samuel F.B. Morse Medal, awarded for excellence in physics (2010)
	% \item Sigma Pi Sigma, National Physics Honors Society (2009--2016)
	% \item George Granger Brown Scholarship (2009)
	\end{itemize}

\section*{Recent presentations}

\begin{itemize}
    \item \emph{Membership modeling and probabilistic modeling with JAX and numpyro}, Streams22 meeting, Carnegie Observatories, 2022 (plenary)
    \item \emph{LSSTC DSFP lecturer in Hierarchical Modeling}, Northwestern U, 2022 (lecture series)
    \item \emph{Mapping Dark Matter with Stellar Streams}, SCSU, 2021 (colloquium)
    \item \emph{A New Era for Galactic Dynamics in the Milky Way}, AAS 237, 2021 (\textbf{invited plenary})
    \item \emph{Mapping Dark Matter with Stellar Streams}, University of Utah, 2020 (colloquium)
    \item \emph{Mapping Dark Matter with Stellar Streams: Signatures of Dark Matter Substructure}, CCPP, New York University, 2020 (colloquium)
    \item \emph{Mapping Dark Matter with Stellar Streams: Imprints of Galactic Dynamical Phenomena}, CCA, Flatiron Institute, 2020 (colloquium)
    \item \emph{Mapping Dark Matter with Stellar Streams}, LSA, University of Michigan, 2020 (colloquium)
    \item \emph{Discovery and characterization of a recent star formation event in the Magellanic Leading Arm}, AAS, Honolulu, 2020 (contributed talk and press conference)
    \item \emph{Discovery and characterization of a recent star formation event in the Magellanic Leading Arm}, A synoptic view of the Magellanic Clouds, ESO Garching 2019 (contributed talk)
    \item \emph{A detailed look at the GD-1 stellar stream}, KITP, Santa Barbara, 2019 (contributed talk)
    \item \emph{The Milky Way as a benchmark}, UCONN, Connecticut, 2019 (colloquium)
    \item \emph{The Milky Way as a benchmark}, Princeton/IAS, Princeton, 2019(colloquium)
    \item \emph{The GD-1 stream and dark matter around the Milky Way}, AAS, Seattle, 2019 (contributed talk)
    \item \emph{The Dynamic Milky Way in the Gaia Era}, University of Arizona, Arizona, 2018 (colloquium)
    \item \emph{The Dynamic Milky Way in the Gaia Era}, Princeton/IAS, Princeton, 2018 (colloquium)
    \item \emph{A disk origin for inner stellar halo structures}, Stellar halos, Heidelberg, 2018 (contributed talk)
    \item \emph{An Overview of the Astropy Project}, Python in Astronomy, NYC, 2018 (invited keynote)
    \item \emph{Binary star science with many targets, few epochs}, SnowPAC, Utah, 2018 (conference)
    \item \emph{The Galactic bar and its effect on stellar streams}, University of Kentucky, 2018 (seminar)
    \item \emph{Comoving stars in the Gaia era}, HAA, NRC-Herzberg, 2018 (seminar)
    \item \emph{Comoving stars in the Gaia era}, University of British Columbia, 2018 (colloquium)
    % \item \emph{Comoving stars in the Gaia era}, DIRAC Institute, UW 2018 (seminar)
    % \item \emph{Improving Gaia parallaxes with data driven models of stars}, Princeton Physics, 2017 (seminar)
    % \item \emph{Comoving stars in Gaia DR1}, Rutgers, 2017 (seminar)
    % \item \emph{The Galactic bar and its effect on stellar streams}, Princeton, 2017 (seminar)
    % \item \emph{The Galactic bar and its effect on stellar streams}, STScI, 2017 (seminar)
    \item \emph{Fitting a straight line to data}, Computational Physics Workshop, Princeton, 2017 (invited)
    % \item \emph{Chaos, stellar streams, and the Galactic matter distribution}, University of Michigan, 2016 (invited seminar)
    % \item \emph{Chaos, stellar streams, and the Galactic matter distribution}, IAS, 2016 (invited seminar)
    % \item \emph{Chaos, stellar streams, and the Galactic matter distribution}, University of Delaware, 2016 (invited seminar)
    % \item \emph{Chaos and Stellar Streams}, AAS 227, 2016 (dissertation talk)
    % \item \emph{Software testing}, AAS 227, 2016 (invited talk)
    % \item \emph{Tidal streams in triaxial systems}, Price Prize Lecture, the Ohio State University, 2015 (invited talk)
	% \item \emph{Tidal streams in triaxial systems}, Stellar streams in the local universe, Ringberg Castle, 2015 (contributed talk)
 %    \item \emph{Inferring the gravitational potential of the Milky Way with a few precisely measured stars}, Local Group Astrostatistics, University of Michigan, 2015 (contributed talk)
 %    \item \emph{Tidal streams in triaxial potentials}, Galaxy lunch, Yale University, 2015 (seminar)
	% \item \emph{Modeling tidal streams and weighing the Milky Way}, Tea talk, Caltech, 2015 (seminar)
	% \item \emph{Tidal streams in triaxial systems}, AAS 225, 2015 (poster)
	% \item \emph{Rewinder}, 2014, Gaia Data Challenge, MPIA (contributed talk)
	% \item \emph{Bayesian statistics}, 2014, Course lecture, Statistics and machine learning in astronomy, Columbia University (lecture)
	% \item \emph{Angle-action coordinates}, 2014, Galaxies lunch, Columbia University (lecture)
	% \item \emph{The potential of the Milky Way}, 2014, Galaxy lunch, Yale University (seminar)
	% \item \emph{Spitzer, Gaia, and the potential of the Milky Way}, 2014, AAS 223 (poster)
	% \item \emph{Probing the Galactic potential with 6D information}, 2013, Gaia Data Challenge, University of Surrey (contributed talk)

\end{itemize}


\section*{Open source development \href{https://github.com/adrn}{\faGithub}}

\begin{itemize}

	\item Core contributor and member of the Coordinating Committee for the \href{http://www.astropy.org/}{Astropy project}
	\item Core developer of
        \href{http://gala.adrian.pw}{gala},
        \href{https://github.com/adrn/thejoker}{thejoker},
        \href{https://github.com/adrn/schwimmbad}{schwimmbad},
        \href{https://github.com/adrn/pyia}{pyia}

\end{itemize}


\section*{Advising and mentorship}

% \begin{itemize}
%     \item \textit{Princeton undergraduates}: Bethlee Lindor (2017), Samuel Moore (2018)
%     \item \textit{Columbia undergraduates}: Tze P. Goh (2014--2015), Adrian Meyers (2014--2015), Kate Steiner (2020)
%     \item \textit{Google Summer of Code}: Manan Agarwal (2015), Jazmin Berlanga (2015), Brett Morris (2015)
%     \item \textit{Summer students}: Cameron Jackson (NSBP Scholar, 2020)
%     \item \textit{Post-baccalaureate}: Kiyan Tavangar (CCA/Flatiron visitor, 2021--2022)
%     \item \textit{CCA Pre-doctoral Program}: Karl Jaehnig (Vanderbilt, 2021), Sophia Lilleengen (U Surrey, 2022)
%     \item \textit{Graduate students}: Semyeong Oh (Princeton, 2016--2018; now postdoc at Cambridge), Tomer Yavetz (Columbia, 2018--), Nora Shipp (U Chicago, 2019--2020; now postdoc at MIT), Nico Garavito-Camargo (U Arizona, 2019--2021; now postdoc at Flatiron Institute), Suroor Gandhi (NYU, 2019--), Katie Chamberlain (U Arizona, 2021--), Juan Guerra (Yale, 2021--), Micah Oeur (UC Merced, 2021--), Alex Riley (Texas A\&M, 2021--)
% \end{itemize}

    % Princeton undergraduates:
    %     - Bethlee Lindor (2017)
    %     - Samuel Moore (2018)

    % Columbia undergraduates:
    %     - Tze P. Goh (2014--2015)
    %     - Adrian Meyers (2014--2015)
    %     - Kate Steiner (2020)

    % Google Summer of Code:
    %     - Manan Agarwal (2015)
    %     - Jazmin Berlanga (2015)
    %     - Brett Morris (2015)

    % Summer students:
    %     -Cameron Jackson (NSBP Scholar, 2020)

    % Post-baccalaureate:
    %     - Kiyan Tavangar (CCA/Flatiron visitor, 2021--2022)

    % CCA Pre-doctoral Program:
    %     - Karl Jaehnig (Vanderbilt, 2021)
    %     - Sophia Lilleengen (U Surrey, 2022)

    % Graduate students:
    %     - Semyeong Oh (Princeton, 2016--2018; now postdoc at Cambridge)
    %     - Tomer Yavetz (Columbia, 2018--)
    %     - Nora Shipp (U Chicago, 2019--2020; now postdoc at MIT)
    %     - Nico Garavito-Camargo (U Arizona, 2019--2021; now postdoc at CCA)
    %     - Suroor Gandhi (NYU, 2019--)
    %     - Katie Chamberlain (U Arizona, 2021--)
    %     - Juan Guerra (Yale, 2021--)
    %     - Micah Oeur (UC Merced, 2021--)
    %     - Alex Riley (Texas A\&M, 2021--)

    % Postdocs:
    %     - Nico Garavito-Camargo
    %     - Sam Grunblatt
    %     - Sarah Pearson
    %     - Jason Hunt


\begin{itemize}
    \item Direct mentor for 9 undergraduate students or \textit{Google Summer of Code} participants (through the Astropy project)
    \item Mentor for 14 graduate students or post-baccalaureate students (resulting in 7 refereed publications; 6 publications in prep.)
    \item Mentor for 4 post-doctoral fellows (resulting in 4 refereed publications; 4 publications in prep.)
\end{itemize}

\section*{Teaching}

\begin{itemize}
    \item Lecturer for the \textit{LSSTC: Data Science Fellowship Program}, 2022, Northwestern U
    \item Workshop coordinator for the \textit{Big Apple Dynamics School}, 2021, Flatiron Institute
    \item Lecturer, breakout leader, participant at Astro Hack Week (2014--2018)
    \item \emph{Data science seminar}, co-organized with Peter Melchior, 2018, Princeton University
    \item \emph{PHY121: Intro to Astronomy}, Prison Teaching Initiative, Fort Dix Correctional Facility
	\item \emph{AST 542: Statistics and Machine Learning}, Co-instructor, 2017, Princeton University
    \item \emph{Galaxies}, Teaching assistant, 2014, Columbia University
	\item \emph{Stars, Planets, and Galaxies}, Lab instructor, 2013, Columbia University
	\item \emph{Earth, Moon, and Planets}, Lab instructor, 2012, Columbia University
	\item \emph{Stars, Planets, and Galaxies}, Teaching assistant, 2012, Columbia University
\end{itemize}

\section*{Workshop and meeting organization}

\begin{itemize}
    \item Chair of organizing committee for \textit{Future of Astronomical Data Infrastructure} at the Simons Foundation, Feb 2023
    \item Organizer of \href{https://stellarstreams.org/streams22}{Streams22: Community Atlas of Tidal Streams}, Nov 2022
    \item Organizer of \textit{DDA 53} meeting at the Flatiron Institute, Apr 2022
    \item Organizer of \href{https://code.astrodata.nyc/}{\textit{Astronomical Software Development}} at the Flatiron Institute, May 2022
    \item Organizer of \href{https://indico.flatironinstitute.org/event/2777/}{\textit{From Data to Software to Science with the Rubin Observatory LSST}} at the Flatiron Institute, Mar 2022
    \item Co-lead organizer of \textit{Big Apple Dynamics School}, a Galactic Dynamics summer school at the Flatiron Institute, June--August 2021
    \item Co-organizer of \textit{Streams21} meeting \href{https://stellarstreams.org/streams21}{Streams21: Constraints on Dark Matter}, Feb 2021
    \item Co-organizer of \href{http://galacticdynamics.nyc/}{Applied Galactic Dynamics Summer School}, postponed until 2021 at earliest
    \item Co-organizer of the \href{http://gaia.lol}{Gaia sprints}, 2016--present
    \item Instructor (Astropy) at \href{http://pydata.org/nyc2017}{PyData NYC}, 2017
    \item Instructor (Machine Learning) at \href{http://astrohackweek.org}{AstroHackWeek}, 2017
    \item Co-organizer of \href{http://scicoder.org}{SciCoder workshop}, 2011--2013, 2015
    \item \href{https://groups.google.com/forum/#!forum/astrohackny}{AstroHackNY}, NYC astronomy \& statistics group meetings, (organizer, 2014-2015)
    \item \href{https://github.com/adrn/nycastroml}{NYCastroML}, machine learning and statistics group meetings, (co-organizer, 2013-2014)
    % \item \emph{Scicoder@AAS}, workshop instructor and co-organizer, AAS 223, Washington, DC, 2014
    % \item \emph{Scicoder@AAS}, workshop instructor and co-organizer, AAS 221, Long Beach, CA, 2013
\end{itemize}

\section*{Public outreach}

\begin{itemize}
    \item Volunteer with the Prison Teaching Initiative, 2017
    \item \emph{The bar at the center of the Galaxy}, 2016, public outreach talk, Astronomy on Tap, NYC
    \item \emph{Galactic synthesizers}, 2015, public outreach talk, Columbia University, NYC
    \item \emph{Dark matter}, 2015, public outreach talk, \href{http://silentbarn.org/2015/03/100-outer-space-party}{100\% Outer Space}, Silent Barn, Brooklyn, NY
    \item Organizer for \href{http://astronomyontap.org/}{Astronomy on Tap} (uptown), 2013-2014, public outreach talks at bars in NYC
    \item \emph{Light}, 2012, public outreach talk for middle school girls, \href{http://www.newstimes.com/news/article/Astronomer-Shoot-for-the-stars-3380793.php}{astro4girls}, Ridgefield Library
    \item Member of \href{http://rv.astro.columbia.edu}{Rooftop variables}, 2011--2016, Isaac E. Young Middle School, New Rochelle, NY (partner teacher: Scott Misner)
    \item Roof captain and manager, 2011--2016, bi-weekly events for \href{http://outreach.astro.columbia.edu/}{Columbia Astronomy outreach}
\end{itemize}

\section*{Professional services \& activities}

\begin{itemize}
	\item Referee: MNRAS, ApJ, A\&A, Phys. Rev. L, Phys. Rev. D, Journal of Open Source Software
	\item Member: AAS, NSBP, NYAS
    \item TACs: NASA, NOAO
\end{itemize}

\end{document}
