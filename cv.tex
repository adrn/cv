\documentclass[12pt,letterpaper]{article}

\usepackage{hyperref}
\usepackage{geometry}
\usepackage[T1]{fontenc}
\usepackage{natbib}

% name here
\def\name{\textbf{Adrian M. Price-Whelan}}

% Date formatting.
\usepackage[yyyymmdd]{datetime}
\renewcommand{\dateseparator}{-}

% ADS query link
\def\adsurl{http://adsabs.harvard.edu/cgi-bin/nph-abs_connect?db_key=AST&db_key=PRE&qform=AST&arxiv_sel=astro-ph&arxiv_sel=cond-mat&arxiv_sel=cs&arxiv_sel=gr-qc&arxiv_sel=hep-ex&arxiv_sel=hep-lat&arxiv_sel=hep-ph&arxiv_sel=hep-th&arxiv_sel=math&arxiv_sel=math-ph&arxiv_sel=nlin&arxiv_sel=nucl-ex&arxiv_sel=nucl-th&arxiv_sel=physics&arxiv_sel=quant-ph&arxiv_sel=q-bio&sim_query=YES&ned_query=YES&adsobj_query=YES&aut_logic=OR&obj_logic=OR&author=price-whelan\%2C+Adrian&object=&start_mon=&start_year=&end_mon=&end_year=&ttl_logic=OR&title=&txt_logic=OR&text=&nr_to_return=200&start_nr=1&jou_pick=ALL&ref_stems=&data_and=ALL&group_and=ALL&start_entry_day=&start_entry_mon=&start_entry_year=&end_entry_day=&end_entry_mon=&end_entry_year=&min_score=&sort=SCORE&data_type=SHORT&aut_syn=YES&ttl_syn=YES&txt_syn=YES&aut_wt=1.0&obj_wt=1.0&ttl_wt=0.3&txt_wt=3.0&aut_wgt=YES&obj_wgt=YES&ttl_wgt=YES&txt_wgt=YES&ttl_sco=YES&txt_sco=YES&version=1}

% PDF metadata
\hypersetup{
  colorlinks = true,
  urlcolor = [rgb]{0.1,0.25,0.5},
  pdfauthor = {\name},
  pdfkeywords = {astrophysics, astronomy, physics},
  pdftitle = {\name: Curriculum Vitae},
  pdfsubject = {Curriculum Vitae},
  pdfpagemode = UseNone
}

% page size
\geometry{
  body={6.5in, 9.0in},
  left=1.0in,
  top=1.0in
}

% text formatting
\usepackage{color}
\definecolor{grey}{gray}{0.5}
\newcommand{\deemph}[1]{\textcolor{grey}{\footnotesize{#1}}}

% heading / footing
\usepackage{fancyheadings}
\pagestyle{fancy}
\renewcommand{\headrulewidth}{0pt}
\lhead{\deemph{Adrian M. Price-Whelan}}
\chead{\deemph{Curriculum Vitae}}
\rhead{\deemph{\thepage}}
\cfoot{\deemph{Last updated: \today}}

% Don't indent paragraphs.
\setlength\parindent{0em}

% Make lists without bullets and compact spacing
\renewenvironment{itemize}{
  \begin{list}{}{
    \setlength{\leftmargin}{1em}
    \setlength{\itemsep}{0.em}
    \setlength{\parskip}{0pt}
    \setlength{\parsep}{0.25em}
    \setlength{\itemindent}{-1em}
  }
}{
  \end{list}
}

% Change section font size and spacing
\usepackage{titlesec}
\titleformat{\section}{\normalfont\fontsize{14pt}{0}\bfseries}{\thesection}{}{}
\titleformat{\subsection}{\normalfont\fontsize{12pt}{0}\bfseries}{\thesubsection}{}{}
\titlespacing{\section}{0em}{-0.em}{0.25em}
\titlespacing{\subsection}{0.5em}{0.em}{0.25em}

% literature links (thanks @dfm)
\newcommand{\doi}[2]{\emph{\href{http://dx.doi.org/#1}{{#2}}}}
\newcommand{\ads}[2]{\href{http://adsabs.harvard.edu/abs/#1}{{#2}}}
\newcommand{\arxiv}[1]{{\href{http://arxiv.org/abs/#1}{arXiv:{#1}}}}

% Journal names
\newcommand{\aanda}{A\&A}
\newcommand{\aj}{AJ}
\newcommand{\apj}{ApJ}
\newcommand{\apjs}{ApJS}
\newcommand{\apjl}{ApJL}
\newcommand{\pasp}{PASP}
\newcommand{\mnras}{MNRAS}
\newcommand{\mnrasl}{MNRAS Letters}

\begin{document}\thispagestyle{empty}\sloppy\sloppypar

% Name and contact, website, etc.
{\huge \name}

\begin{itemize}
  \item Department of Astronomy, Columbia University, New York, NY 10027
  \item \href{mailto:adrn@astro.columbia.edu}{adrn@astro.columbia.edu} --- 
		\href{http://adrian.pw}{http://adrian.pw}
\end{itemize}

\section*{Education}
	\begin{itemize}
	\item PhD 2016 (expected), Department of Astronomy, Columbia University\\
		{Advisor: K. V. Johnston}
	\item MSc 2014, Department of Astronomy, Columbia University\\
		{Advisor: K. V. Johnston}
	\item BA (honors) 2010, Department of Physics, New York University\\ 
		{Advisor: D. W. Hogg, \emph{Summa cum laude}}
	\end{itemize}

\section*{Honors and Awards}
	\begin{itemize}
	\item NSF Graduate Research Fellowship (2012--2015)
	\item Dean's List, New York University, (2007--2010)
	\item SDSS-III architect (2012--)
	\item Phi Beta Kappa, Beta of New York, (2010--)
	\item Samuel F.B. Morse Medal, awarded for excellence in physics (2010)
	\item Sigma Pi Sigma, National Physics Honors Society (2009--)
	\item George Granger Brown Scholarship (2009)
	\end{itemize}

\section*{Grants and observing }
	\begin{itemize}
	\item {\it The Triangulum-Andromeda stellar clouds: a population of halo stars kicked out of the Galactic disk?}, spectroscopy, Hiltner Telescope, MDM Observatory (PI, 2015)
	\item {\it Spitzer Merger History and Shape of the Galactic Halo}, Spitzer mission (Co-I, 2014-2015)
	\item {\it Gaia, Spitzer, and the potential of the Milky Way}, NASA theory grant (Co-I, 2014-2016)
	\item Sigma Xi Grants in Aid of Research (PI, 2013-2014)
	\item {\it Probing the Milky Way's dark matter halo with RR Lyraes}, spectroscopy, Hiltner Telescope, MDM Observatory (PI, 2013)
	\end{itemize}

\section*{Recent Publications (\href{\adsurl}{ADS})}
	\subsection*{First-author}
	\begin{itemize}

\item {\bf Price-Whelan, A. M.}, Johnston, K. V. et al., 2015,
    \doi{10.1093/mnras/stv1324}{A re-interpretation of the Triangulum-Andromeda stellar clouds: a population of halo stars kicked out of the Galactic disc},
    \mnras, 452, 676 (\arxiv{1503.08780})

\item {\bf Price-Whelan, A. M.}, Hogg, D. W., Johnston, K. V., Hendel, D., 2014,
    \doi{10.1088/0004-637X/794/1/4}
    {Inferring the Gravitational Potential of the Milky Way with a Few Precisely Measured Stars},
    \apj, 794, 4 (\arxiv{1405.6721})
    
\item {\bf Price-Whelan, A. M.}, Ag\"ueros, M. A., et al., 2014,
    \doi{10.1088/0004-637X/781/1/35}
    {Statistical Searches for Microlensing Events in Large, Non-uniformly Sampled Time-Domain Surveys: A Test Using Palomar Transient Factory Data},
    \apj, 781, 35 (\arxiv{1311.3683})
    
\item {\bf Price-Whelan, A. M.}, Johnston, K. V., Hogg, D. W., 2013,
    \doi{10.1088/2041-8205/778/1/L12}
    {Spitzer, Gaia, and the Potential of the Milky Way},
    \apjl, 778, L12 (\arxiv{1308.2670})

	\end{itemize}

	\subsection*{Contributed}

	\begin{itemize}
	
\item Charisi, M., Bartos, I., Haiman, Z., {\bf Price-Whelan, A. M.}, M\'arka, S., 2015,
    \emph{Multiple periods in the variability of the supermassive black hole binary candidate quasar PG1302-102?},
    \mnras\ submitted (\arxiv{1502.03113})

\item Pearson, S.; K\"upper, A. H. W., Johnston, K. V., {\bf Price-Whelan, A. M.}, 2015,
    \doi{10.1088/0004-637X/799/1/28}
    {Tidal Stream Morphology as an Indicator of Dark Matter Halo Geometry: The Case of Palomar 5},
    \apj, 799, 28 (\arxiv{1410.3477})
    
\item Alam, S. et al., 2015,
    \emph{The Eleventh and Twelfth Data Releases of the Sloan Digital Sky Survey: Final Data from SDSS-III},
    \apjs\ submitted (\arxiv{1501.00963})
    
\item Andrews, J. J., {\bf Price-Whelan, A. M.}, Ag\"ueros, M. A., 2015,
    \doi{10.1088/2041-8205/797/2/L32}
    {The Mass Distribution of Companions to Low-mass White Dwarfs},
    \apjl, 797, L32 (\arxiv{1412.0114})
    
\item Astropy Collaboration et al., 2013, 
    \doi{10.1051/0004-6361/201322068}
    {Astropy: A community Python package for astronomy},
    \aanda, 558, A33 (\arxiv{1307.6212})
    
	\end{itemize}

\section*{Leadership and Advising}
	\begin{itemize}
	\item \href{https://groups.google.com/forum/#!forum/astrohackny}{AstroHackNY}, NYC astronomy \& statistics group meetings, (organizer, 2014-2015)
	\item \href{https://github.com/adrn/nycastroml}{NYCastroML}, machine learning and statistics group meetings, (co-organizer, 2013-2014)
	\item Undergraduate students: Adrian Meyers (senior thesis, 2014-2015, now graduate student at Yale), Jazmin Berlanga (Google Summer of Code, 2015)
	\end{itemize}

\section*{Recent Presentations}

\begin{itemize}
	% \item \emph{Tidal streams in triaxial systems}, 2015, Streams..., Ringberg Castle, Germany
	\item \emph{Inferring the gravitational potential of the Milky Way with a few precisely measured stars}, 2015, Local Group Astrostatistics, Ann Arbor, MI
	\item \emph{Tidal streams in triaxial systems}, 2015, Galaxy lunch, Yale University, New Haven, CT
	\item \emph{Modeling tidal streams and weighing the Milky Way halo}, 2015, Tea talk, Caltech, Pasadena, CA
	\item \emph{Tidal streams in triaxial systems}, 2015, AAS 225, Seattle, WA
	\item \emph{Rewinder}, 2014, Gaia Data Challenge, MPIA, Heidelberg, Germany
	\item \emph{Bayesian statistics}, 2014, Course lecture, Statistics and machine learning in astronomy, Columbia University, New York, NY
	\item \emph{Angle-action coordinates}, 2014, Lecture, Columbia University, New York, NY
	\item \emph{The potential of the Milky Way}, 2014, Galaxy lunch, Yale University, New Haven, CT
	\item \emph{Spitzer, Gaia, and the potential of the Milky Way}, 2014, AAS 223, Washington, DC
	\item \emph{Probing the Galactic potential with 6D information}, 2013, Gaia Data Challenge, University of Surrey, Surrey, UK

\end{itemize}

\section*{Open Source Development (\href{https://github.com/adrn}{GitHub profile})}
\begin{itemize}
	
	\item Core contributor to the \href{http://www.astropy.org/}{Astropy} project
	\item Maintainer of \href{http://tutorials.astropy.org/}{Astropy Tutorials}
	\item Developer and co-creator of \href{http://d3po.org}{D3PO}
	\item Contributor to \href{http://matplotlib.org/}{matplotlib}, \href{http://dan.iel.fm/emcee/current/}{emcee}, \href{https://github.com/dfm/triangle.py}{triangle.py}

\end{itemize}

\section*{Workshop Organization}
\begin{itemize}
	\item Co-organizer of \href{http://scicoder.org}{SciCoder workshop}, 2011--2013, 2015
	\item \emph{Scicoder@AAS}, workshop instructor and co-organizer, 2014, AAS 223, Washington, DC
	\item \emph{Scicoder@AAS}, workshop instructor and co-organizer, 2013, AAS 221, Long Beach, CA
\end{itemize}

\section*{Public Outreach}

\begin{itemize}
	\item \emph{Galactic synthesizers}, 2015, public outreach talk, Columbia University, New York, NY
	\item \emph{Dark matters}, 2015, public outreach talk, \href{http://silentbarn.org/2015/03/100-outer-space-party}{100\% Outer Space}, Silent Barn, Brooklyn, NY
	\item Organizer for \href{http://astronomyontap.org/}{Astronomy on Tap} (uptown), 2013-2014, public outreach talk series at bars in NYC
	\item \emph{Light}, 2012, public outreach talk for middle school girls, \href{http://www.newstimes.com/news/article/Astronomer-Shoot-for-the-stars-3380793.php}{astro4girls}, Ridgefield Library, Ridgefield, CT
	\item Member of \href{http://rv.astro.columbia.edu}{Rooftop variables}, 2011--present, Isaac E. Young Middle School, New Rochelle, NY (partner teacher: Scott Misner)
	\item Roof captain / volunteer, 2011--present, \href{http://outreach.astro.columbia.edu/}{Columbia Astronomy outreach}

\end{itemize}

\section*{Teaching}
\begin{itemize}
	\item \emph{Stars, Planets, and Galaxies}, Lab instructor, 2013, Columbia University
	\item \emph{Earth, Moon, and Planets}, Lab instructor, 2012, Columbia University
	\item \emph{Stars, Planets, and Galaxies}, Teaching Assistant, 2012, Columbia University
	\item \emph{Life in the Universe}, Teaching Assistant, 2011, Columbia University
	\item \emph{Classical and Quantum Waves Lab}, Teaching Assistant, 2011, New York University
	\item \emph{Physics III Lab}, Teaching Assistant, 2010, New York University
\end{itemize}

\end{document}
